\section{Proposal Draft}
\begin{itemize}
\item \textbf{First} : Recommend papers to users.
\begin{itemize}
\item \textbf{Topic extraction}: Extract the topic of the paper as labels.
\item \textbf{Paper Rank} : Rank the paper by citation
\begin{itemize}
\item \textbf{Method}: rank by 5 years citation accumulation, if there is papers' citation less than 5 years, predict the citation of the new paper and join the rank.
\end{itemize}
\item \textbf{User analyses}(if we have user group data): Analyses users feature {Gender, Age, Interest area} to decide the label to recommend. 
\item \textbf{Introduction generate}: When the paper have been selected, generate the introduction in a short form(1-2 sentence) for recommend use.
\begin{itemize}
\item \textbf{Method} : Use Attention-Based summarization to generate the summary based on abstract(ABS)(Rush et.al 2015)
\end{itemize}
\end{itemize}
 
\item \textbf{Second} : Reading Helper

Question Understanding: Read the structure of the sentence and classify which kind of answer should return.

Answer Classification:
\begin{itemize}
\item Elements: what's the problem/model/datasets/...

  \textbf{Example}:
  \begin{itemize}
  \item Q: What's the problem of this paper? A:(An abstract of the problem definition)
  \item Q: What model are using in this paper? A:(A sort of model used in this paper)
  \end{itemize}

  \textbf{Method}:Locate the section introducing the element, pick the key word of this section or do an abstract.

\item Where can i find it? 

  \textbf{Example}:
  \begin{itemize}
  \item Q: Where can I find the definition of XXX? A:(Place where XXX is defined).
  \item Q: Where can I find the explanation of Figure 2? A:(Place where figure 2 is explained)
  \end{itemize}

  \textbf{Method}:
  \begin{itemize}
  \item1, Extract the key word user wand to find, and locate all the possible choice in the
paper.
\item2, Rank the possible of the (question,answer) choice use deep learning method to
find which choice should possible be the best choice.(Can use some ``where'' question
example as the train dataset)


  \end{itemize}


\item What's the topic of this paragraph

  \textbf{Example}:
  \begin{itemize}
  \item Q: What is the topic of this paper? A:(Paper topic abstract)
  \item Q: What is this section mainly about A:(Section abstract)

  \end{itemize}

  \textbf{Method}:
    \begin{itemize}
  \item1, Sentence pick method using Lex-rank. Select the top-rate sentence.
  \item2, Sentence generate method using ABS(attention-based summarization).
  \end{itemize}
\item What's meaning of this term (wiki or in article) 

  \textbf{Example}:
  \begin{itemize}
  \item Q: What's the meaning of LSTM? A:(A wiki about LSTM)
  \item Q: What's Memory-LSTM? A:(Memory-LSTM definition in the article)
  \end{itemize}

  \textbf{Method}:
	\begin{itemize}
    \item1,Locate to Problem definition/Model/datasets section
    \item2,Exctract the Key word in this section.
    \end{itemize}
  \begin{itemize}
  \item1, For the keyword in the question, we can use the sentence selection method to search the paper and the related paper include the key word, then use the sentence pick method like Lex-rank or Centroid to decide the survey sentence of this key word.(Jha et.al 2013)
  
  \item2, We can search the key word on Wiki and return the answer if the question word is a general word.
  \end{itemize}
  
\item Recommendations

  \textbf{Example}:
  \begin{itemize}
  \item Q: Any related paper? A: (Paper in related works/ Cite)
  \item Q: Similar paper? A: (Similar paper recommendation)
  \item Q: Author in same research area? A:(Author recommendation)
  
  \end{itemize}

  \textbf{Method}: Link to Aminer to select the answer.
\item User Interface
\end{itemize}
\end{itemize}
